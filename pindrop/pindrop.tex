
\documentclass{article}

\usepackage{fullpage}
\usepackage{amsmath}

\title{Review of `PinDr0p: Using Single-Ended Audio Features to Determine Call Provenance`}
\author{Michael Deakin}

\begin{document}
\maketitle
This paper discusses a signal based method of call provenance.
It makes no assumptions about the accuracy of the information on the source of the phone call,
and instead uses the actual audio signal to identify the types of phone networks the signal traversed.
To do this, distinguishing features between signals which have travelled over specific types of networks are identified.
These are statistical features, so the resulting data on them is then classified by a machine learning algorithm.
The types of networks identified include VoIP with speech encoded with G.711, Speex, or iLBC,
cellular networks encoded with GSM-FR,
and dedicated PSTN networks encoded with G.711.
\section{VoIP Features}
One of the major distinguishing features of VoIP is that packets are frequently lost.
Thus, to detect a VoIP network, it makes sense to look for parts of the signal which are zeroed out,
where other information suggests it shouldn't be.
To do this, the Short Time Average Energy function is used, with a Hamming window function.
\begin{figure}[h]
  \[ E_n=\sum_{m=-\infty}^{\infty}y^2(m) w(n-m) \]
  \caption{Short Time Average Energy}
\end{figure}
\begin{figure}[h]
  \[ w(n)=\alpha - \beta \cos\Big(\frac{2\pi n}{N - 1}\Big) \]
  \caption{Hamming Window Function}
\end{figure}
Since packets contain at least 10 ms of audio represented by 80 samples of speech,
choosing a window length less than 80 samples causes multiple values of $E_n$ can be affected by a dropped packet.
This allows us to detect dropped packets by looking for a falling edge, a floor, and a rising edge.
If any of these characteristics are missing, it's not clear if a packet loss has occured,
and these packet drops cannot be detected with this method.
A test of this method on $15 s$ of audio showed a high correspondence between detected dropped packets and actual dropped packets.\\
Packet drops also tell us the codec used in the VoIP network.
The distinguishing feature is the length of the floor resulting from the dropped packet.
iLBC encodes $30 ms$ per packet, and Speex encodes $20 ms$ per packet.
G.729 encodes $10 ms$ per packet by default, and G.711 encodes $20 ms$ per packet by default;
so it is usually safe to assume these floor lengths correspond to these codec types.
Furthermore, since the case of multiple consecutive packets being dropped is rarer than a single packet being dropped,
the most frequent floor length will have a high correspondence with the codec type.\\
Some VoIP networks use a reactive recovery measure to detect lost packets and try to give a credible output signal.
This is done in one of several ways: Outputting silence, outputting noise,
and performing an interpolation based on previous packets.
The STE method described previously is capable of detecting the first two, with only an adjustment of the signal floor.
Unfortunately, the third method is the most common one, making detection more difficult;
but not impossible since these methods are generally deterministic.
This makes a detection a simple matter of looking at correlations between previous packets and the current one.\\
G.711 performs waveform substitution, repeating part of the previous packet, and is easy to detect.\\
iLBC uses a linear predictive coding method based on the source filter model of speech production.
This method inverse-filters the voiced sounds, leaving behind the original sound energy residual.
iLBC uses the residual, synthesis filters, and a dynamic codebook to compress the audio signal.
When a packet is lost, the decoder uses the residual from the previous signal,
and adds a random excitation to it.
It creates a new pitch synchronous residual {\bf(What is that?)} for the packet to be concealed.\\
To detect PLC in iLBC
{\bf(How do we know it's iLBC at this point? Is this indiscriminantly applied?
  Or is it guessed based on the improbability of other codecs being used?)},
the audio is split into $30 ms$ packets, as that is the default for iLBC.
The pitch synchronous residual is created from each packet and compared to the previous one.
The pitch synchronous residual is not usually highly correlated,
so a high correlation between these packets implies that the current one was lost
{\bf(How much does misalignment of the lost packet and the assumed packet affect this correlation?
  In the worst case, the guessed packets are exactly out of phase with the real packets.
  What is the expected correlation then,
  and how does it compare to the normal amount of correlation between two consecutive packets?
  Or is there a mechanism to detect when this happens and realign the guessed packets?)}.

\end{document}
